\documentclass{beamer}
\definecolor{azulPoP}{rgb}{0.243,0.466,0.701}
\mode<presentation> {
%\usetheme{Warsaw}
\setbeamercolor{frametitle}{fg=white,bg=azulPoP}
\setbeamercolor{subsection in head/foot}{fg=white,bg=black}
\setbeamercolor{block title}{bg=azulPoP,fg=white}
\setbeamercolor{title}{bg=azulPoP,fg=white}
}

\usepackage{graphicx} % Allows including images
\usepackage{booktabs} % Allows the use of \toprule, \midrule and \bottomrule in tables
\usepackage[utf8x]{inputenc}
\usepackage{minted}
\usepackage[portuges]{babel}
%----------------------------------------------------------------------------------------
%	TITLE PAGE
%----------------------------------------------------------------------------------------
\setbeamerfont{footnote}{size=\tiny}
\title[Introdução à Orientação a Objetos I]{Introdução à Orientação a Objetos I} % The short title appears at the bottom of every slide, the full title is only on the title page

\author{Rafael Silva Guimarães} % Your name
\institute[IFES] 
{
Instituto Federal do Espírito Santo \\ % Your institution for the title page
\medskip
\textit{rafaelg@ifes.edu.br} \\ % Your email address
\textit{http://rafaelguimaraes.net} \\
\textit{https://github.com/rafaelsilvag/ifesJava/}
}
\date{\today} % Date, can be changed to a custom date
\logo{\includegraphics[scale=0.2]{imagens/ifes.png}}

\begin{document}

\setbeamertemplate{footline}
{%
\leavevmode%
\hbox{%

%\begin{beamercolorbox}[wd=.20\paperwidth,ht=1.0ex,dp=1.125ex,left]{author
%in head/foot}%
%\usebeamerfont{title in head/foot}\insertshortauthor\hspace{.3cm}
%\end{beamercolorbox}%

\begin{beamercolorbox}[wd=.45\paperwidth,ht=0.8ex,dp=1.125ex,left]{title
in head/foot}%
\usebeamerfont{author in head/foot}\hspace{.3cm}\insertshorttitle

\end{beamercolorbox}%

\begin{beamercolorbox}[wd=.1\paperwidth,ht=0.8ex,dp=1.125ex,center]{title
in head/foot}%
\usebeamerfont{author in head/foot}\insertframenumber/\inserttotalframenumber
\end{beamercolorbox}%
}%
\vskip0pt%
}

\begin{frame}
\titlepage % Print the title page as the first slide
\end{frame}

\begin{frame}
\frametitle{Glossário} % Table of contents slide, comment this block out to remove it
\tableofcontents % Throughout your presentation, if you choose to use \section{} and \subsection{} commands, these will automatically be printed on this slide as an overview of your presentation
\end{frame}

%----------------------------------------------------------------------------------------
%	PRESENTATION SLIDES
%----------------------------------------------------------------------------------------

%------------------------------------------------
\section{Estruturas de Decisão}

\begin{frame}
\frametitle{Estrururas de Decisão}
	\begin{itemize}
		\item Uma estrutura de decisão altera o fluxo de execução do programa. Ele faz isso através de uma estrutura aonde você especifica uma condição para que ocorra tal alteração em um determinado programa.
		\item No Java, as estruturas de decisão possuem syntaxe idêntica da linguagem C, vista por vocês no semestre anterior. As estruturas que você deve ter conhecimento são: \textit{IF} e \textit{SWITCH}.
	\end{itemize}
\end{frame}
\begin{frame}
	\frametitle{Estrururas de Decisão}
	\begin{itemize}
		\item Vejamos um de utilização da estrurura \textit{IF} no algoritmo ExemploIf01.
		\item \textbf{Obs.: Assim como no C, a linguagem Java também é sensível a letras maiúsculas e minísculas.}
	\end{itemize}
\end{frame}
\begin{frame}
	\frametitle{Estrururas de Decisão}
	\begin{example}[ExemploIf01.java]
		\inputminted[tabsize=2, fontsize=\footnotesize]{java}{codigos/ExemploIf01.java}
	\end{example}
\end{frame}
\begin{frame}
	\frametitle{Estrururas de Decisão}
	\begin{itemize}
		\item Estamos alterando a forma de como iremos exibir e capturar as informações para o usuário. Estaremos utilizando uma classe gráfica que exibe e coleta informações do usuário chamada de "\textit{JOptionPane}".
		\item Esta classe faz parte do pacote \textit{Swing}, responsável por armazenar todas as classes de manipulação gráfica no Java. Iremos aprofundar melhor nela em um outro momento em nossa disciplina.
	\end{itemize}
\end{frame}
\begin{frame}
	\frametitle{Estrururas de Decisão}
	\begin{itemize}
		\item Lembre-se! Para capturar e exibir os dados utilizaremos conforme podemos observar no código abaixo:
		\item JOptionPane.showMessageDialog(null, "Valor");
			\begin{itemize}
				\inputminted[tabsize=2, fontsize=\footnotesize]{java}{codigos/ExemploIf02.java}
			\end{itemize}
		\item JOptionPane.showInputDialog("Informe um valor: ");
			\begin{itemize}
				\inputminted[tabsize=2, fontsize=\footnotesize]{java}{codigos/ExemploIf03.java}
			\end{itemize}
	\end{itemize}
\end{frame}
\begin{frame}
	\frametitle{Estrururas de Decisão}
	\begin{itemize}
		\item Algumas comparações serão alteradas. Por exemplo, em C para compararmos strings era necessário utilizarmos uma coleção de funções na biblioteca "string.h". No caso do Java, uma String na verdade é um objetos que contém métodos e atributos. Veja no exemplo a seguir:
		\item Podemo observar que no objeto do tipo String temos um método chamado de "\textit{equal}" ou "\textit{equalIgnoreCase}", responsável por comparar 2 strings. Portanto, o método irá retornar 2 valores, "true" ou "false".
	\end{itemize}
\end{frame}
\begin{frame}
	\frametitle{Estrururas de Decisão}
	\begin{example}[ExemploIf04.java]
		\inputminted[tabsize=1, fontsize=\footnotesize]{java}{codigos/ExemploIf04.java}
	\end{example}
\end{frame}
\begin{frame}
	\frametitle{Estrururas de Decisão}
	\begin{example}[ExemploIf05.java]
		\inputminted[tabsize=1, fontsize=\footnotesize]{java}{codigos/ExemploIf05.java}
	\end{example}
\end{frame}


\begin{frame}
	\frametitle{Estrururas de Decisão}
	\begin{itemize}
		\item Vamos ver um exemplo de utilização da estrutura de decisão Switch em Java, no algoritmo ExemploSwitch01.
		\item Nesta etapa executamos uma "Conversão de Tipos", passando um valor do tipo String para um valor do tipo Integer.
	\end{itemize}
	\begin{table}
		\begin{tabular}{cl l}
			\toprule
			\textbf{Valor Inicial} & \textbf{Tipagem Final} & \textbf{Formato para conversão} \\
			\midrule
			int x = 10 & float & float y = (float) x  \\
			int x = 10 & double & double y = (double) x  \\
			float x = 10.5 & int & int y = (int) x  \\
			\bottomrule
		\end{tabular}
		%\caption{Tipos Primitivos em Java}
	\end{table}
\end{frame}
\begin{frame}
	\frametitle{Estrururas de Decisão}
	\inputminted[tabsize=1, fontsize=\footnotesize]{java}{codigos/ExemploSwitch01.java}
\end{frame}
\begin{frame}
	\frametitle{Estrururas de Decisão}
	\begin{table}
		\begin{tabular}{cl l}
			\toprule
			\textbf{Valor Inicial} & \textbf{Tipagem Final} & \textbf{Formato para conversão} \\
			\midrule
			String x = "10" & int & int y = Integer.parseInt(x)  \\
			String x = "10.50" & float & float y = Float.parseFloat(x)  \\
			String x = "10.50" & double & double y = Double.parseDouble(x)  \\
			String x = "IFES" & Vetor de Bytes & byte b[] = x.getBytes()  \\
			int x = 10 & String & String y = String.valueOf(x) \\
			float x = 10.5 & String & String y = String.valueOf(x) \\
			double x = 235.222 & String & String y = String.valueOf(x) \\
			byte x[] & String & String y = new String(x) \\
			\bottomrule
		\end{tabular}
		%\caption{Tipos Primitivos em Java}
	\end{table}
\end{frame}


%-------------------------------------
\section{Estruturas de Repetição}

\begin{frame}
	\frametitle{Estruturas de Repetição}
	\begin{itemize}
		\item Uma estrutura de repetição, cada ciclo de repetição é chamado de iteração.
		\item No Java, as estruturas de repetição possuem syntaxe idêntica da linguagem C, vista por vocês no semestre anterior. As estruturas que você deve ter conhecimento são: \textit{FOR}, \textit{FOREACH}, \textit{WHILE} e \textit{DO...WHILE}.
	\end{itemize}
\end{frame}
\begin{frame}
	\frametitle{Estruturas de Repetição - FOR }
	\begin{example}[ExemploFor01.java]
		\inputminted[tabsize=1, fontsize=\footnotesize]{java}{codigos/ExemploFor01.java}
	\end{example}
\end{frame}
\begin{frame}
	\frametitle{Estruturas de Repetição - FOREACH }
	\begin{example}[ExemploFor02.java]
		\inputminted[tabsize=1, fontsize=\footnotesize]{java}{codigos/ExemploFor02.java}
	\end{example}
\end{frame}
\begin{frame}
	\frametitle{Estruturas de Repetição - WHILE }
	\begin{example}[ExemploFor02.java]
		\inputminted[tabsize=1, fontsize=\footnotesize]{java}{codigos/ExemploWhile01.java}
	\end{example}
\end{frame}
\begin{frame}
	\frametitle{Estruturas de Repetição - DO...WHILE }
	\inputminted[tabsize=1, fontsize=\footnotesize]{java}{codigos/ExemploWhile02.java}
\end{frame}
%\begin{frame}
%	\frametitle{Tipos de Dados}
%	\begin{example}[TiposDados01.java]
%		\inputminted{java}{codigos/TiposDados01.java}
%	\end{example}
%\end{frame}

%------------------------------------------------
\section{Referências}

\begin{frame}
	\frametitle{Referências}
	\footnotesize{
		\begin{thebibliography}{99} % Beamer does not support BibTeX so references must be inserted manually as below
			\bibitem[Cornell, G. ; Horstmann, S. C., 2003]{p1} Cornell, G. ; Horstmann, S. C. 
			\newblock Core Java 2: Fundamentos (vol.1.)
			\newblock \emph{Pearson Makron Books} São Paulo
		\end{thebibliography}
		\begin{thebibliography}{99} % Beamer does not support BibTeX so references must be inserted manually as below
			\bibitem[Cornell, G. ; Horstmann, S. C., 2003]{p2} Cornell, G. ; Horstmann, S. C. 
			\newblock Core Java 2: Recursos Avançados (vol.2.)
			\newblock \emph{Pearson Makron Books} São Paulo
		\end{thebibliography}
		\begin{thebibliography}{99} % Beamer does not support BibTeX so references must be inserted manually as below
			\bibitem[Sebesta, R.W., 2003]{p3} Sebesta, R.W.
			\newblock Conceitos de Linguagens de Programação
			\newblock \emph{Bookman} São Paulo
		\end{thebibliography}
		\begin{thebibliography}{99} % Beamer does not support BibTeX so references must be inserted manually as below
			\bibitem[Deitel, Paul; Deitel, Harvey, 2010]{p4} Deitel, Paul; Deitel, Harvey
			\newblock Java - Como Programar
			\newblock \emph{Pearson} São Paulo
		\end{thebibliography}
	}
\end{frame}

%----------------------------------------------------------------------------------------

\end{document} 
